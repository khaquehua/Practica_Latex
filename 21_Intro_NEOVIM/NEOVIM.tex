\documentclass{article}

\begin{document}
 
\begin{center}
  Holas de vim
\end{center}

Hola de vim

hola

Hola hola

Texto para copiar

Hola Kevin para cortar y pegar

Texto para copiar
Hola Kevin para cortar y pegar

\begin{center}
  Hola de vim
\end{center}

Hola hola

Texto para copiar

Hola Kevin para cortar y pegar
\textbf{Para atajos}

\begin{itemize}
  \item Para guardar: ctrl + s (control save)
  \item Para retroceder ctrl + z: en modo normal tecla u.
  \item Para avanzar: ctrl + r
  \item Para buscar usar /
  \item Para desglozar los encapsulados rapidamente en uno zr
  \item Para crear un nuevo archivo o carpeta, se colocar en el buscardor ``e'' la letra m
  \item Para localizar el contenido de un capitulo se pone en la parte y despues ``ld''
  \item v: Modo visual
  \begin{itemize}
    \item Para cortar una linea de codigo seleccionamos en modo visual y luego se pulsa d
    \item Para copiar igualmente seleccionamos y se pula y
  \end{itemize}
  \item espacio: Modo normal escape
    \begin{itemize}
      \item Para cortar: dd
      \item Para pegar: p
      \item Para copiar: yy
    \end{itemize}
  \item i: Modo editor o insertar
  \item dos puntos: indica modo comandos
   \begin{itemize}
     \item Si colocas :VimCompile te compila el documento
   \end{itemize}
 \item Para navegar en ventana: ctrl + w, las ventanas son denominadas buffers

\end{itemize}

Al invocar un snippet como table usando la tecla se puede desplazar a las diferentes opciones del table como tipo, despues a caption, despues a label y finalmente al relleno de la tabla


\end{document}
