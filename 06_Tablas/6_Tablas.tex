\documentclass{article}
\usepackage[paperwidth=10cm,paperheight=4cm, margin=1mm]{geometry}
\usepackage{float}
\begin{document}
\begin{table}
\centering
\begin{tabular}{cc}
Nombre & Puntaje \\
Alan & 100 \\
Bob & 60 \\
Chris & 10 \\
David & 80
\end{tabular}
\caption{}
\end{table}

\begin{tabular}{|cc|c|}
\hline
p & q & p $\wedge$ q\\
\hline
v & v & v\\
v & f & f\\
f & v & f\\
f & f & f\\
\hline
\end{tabular}

\begin{table}[H]
\centering
\begin{tabular}{|cc|c|}
\hline
p & q & p $\wedge$ q\\
\hline
v & v & v\\
v & f & f\\
f & v & f\\
f & f & f\\
\hline
\end{tabular}
\caption{Tabla de prueba}
\label{tab:tablaVerdad}
\end{table}
En la tabla \ref{tab:tablaVerdad} se muestra la tabla de verdad.

\begin{tabular}{|c|c|c|}
\hline
$p$ & $q$ & $p \rightarrow q$ \\ \hline
0 & 0 & 1 \\
0 & 1 & 1 \\ \cline{1-2}
1 & 0 & 0 \\
1 & 1 & 1 \\ \hline
\end{tabular}

\begin{tabular*}{5cm}{|cc|c|}
\hline
p & q & p $\wedge$ q\\
\hline
V & V & V\\
V & F & F\\
F & V & F\\
F & F & F\\
\hline
\end{tabular*}

Para obtener una columna en negrilla

\begin{tabular}{|p{1cm}|p{1cm}|p{1cm}|}
\hline
a a a a a a a a a a a a a a a a & b b b b b b b b b b
,! b & c c c c \\
\hline
\end{tabular}




\end{document}

